% From first drafts of Intro

%Although technology offers many tools and opportunities, its mere presence does not assure enhanced educational experiences. Thoughtful integration is essential, as research of digital technologies in education indicates that their effectiveness is often tied more to the quality of design and implementation, rather than just availability for use \parencite{higgins_impact_2012}. Similar observations have been made for interventions involving AI \parencite{holmes_state_2022}. A deeper understanding of student-technology interactions its necessary to guide the design of improved educational tools and foster better learning experiences. 

%Yet, within both the AIEd body of research there is a notable absence of educational specialists representation, a scarcity of qualitative and mixed-methods studies, and a limited variety in data gathering approaches. Particularly relevant is the lack of experiences in physical learning environments and objectives that extend beyond just enhancing students' academic performance \parencite{zawacki-richter_systematic_2019, holmes_state_2022, grover_computational_2013}. 

%Unfortunately, empirical studies on LLMs' impact in educational environments remain sparse \parencite{montenegro-rueda_impact_2023, lo_what_2023}. Since the launch of "ChatGPT" (based on GPT-3) earlier this year, there has been a rapid growth in reported educational uses and impacts, though they remain somewhat limited in scope. Predominantly, existing research either evaluates GPT's performance on traditional assessment methods \parencite{nisar_is_2023}  or delves into theoretical exploration of its potential opportunities and challenges \parencite{cain_gpteammate_2023} . Only a select few of them have explored empirical outcomes of granting students unrestricted use of LLMs in their learning process, with the results appearing optimistic \parencite{fauzi_analysing_2023, yilmaz_effect_2023}. More broadly, literature review in AIEd  underscores the necessity for more hands-on educational research approaches and a wider array of analytical methods and data gathering techniques \parencite{zawacki-richter_systematic_2019, holmes_state_2022}. Notably, a significant gap exists within physical educational spaces, as we could not find empirical studies that specifically crafted a methodical design or purposeful approach for LLMs' integration.
