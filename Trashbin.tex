% From first drafts of Intro

%Although technology offers many tools and opportunities, its mere presence does not assure enhanced educational experiences. Thoughtful integration is essential, as research of digital technologies in education indicates that their effectiveness is often tied more to the quality of design and implementation, rather than just availability for use \parencite{higgins_impact_2012}. Similar observations have been made for interventions involving AI \parencite{holmes_state_2022}. A deeper understanding of student-technology interactions its necessary to guide the design of improved educational tools and foster better learning experiences. 

%Yet, within both the AIEd body of research there is a notable absence of educational specialists representation, a scarcity of qualitative and mixed-methods studies, and a limited variety in data gathering approaches. Particularly relevant is the lack of experiences in physical learning environments and objectives that extend beyond just enhancing students' academic performance \parencite{zawacki-richter_systematic_2019, holmes_state_2022, grover_computational_2013}. 

%Unfortunately, empirical studies on LLMs' impact in educational environments remain sparse \parencite{montenegro-rueda_impact_2023, lo_what_2023}. Since the launch of "ChatGPT" (based on GPT-3) earlier this year, there has been a rapid growth in reported educational uses and impacts, though they remain somewhat limited in scope. Predominantly, existing research either evaluates GPT's performance on traditional assessment methods \parencite{nisar_is_2023}  or delves into theoretical exploration of its potential opportunities and challenges \parencite{cain_gpteammate_2023} . Only a select few of them have explored empirical outcomes of granting students unrestricted use of LLMs in their learning process, with the results appearing optimistic \parencite{fauzi_analysing_2023, yilmaz_effect_2023}. More broadly, literature review in AIEd  underscores the necessity for more hands-on educational research approaches and a wider array of analytical methods and data gathering techniques \parencite{zawacki-richter_systematic_2019, holmes_state_2022}. Notably, a significant gap exists within physical educational spaces, as we could not find empirical studies that specifically crafted a methodical design or purposeful approach for LLMs' integration.

%GPT (Generative Pre-Trained) language models are a particular class of AI systems. They are able to perform a variety of tasks, from translation to text summarization and question answering, often surpassing than human-average performance \parencite{srivastava_beyond_2022}.  As the volume of training data and trained model parameters has increased, their adaptability to tasks beyond their initial training scope has expanded impressively. Current benchmarks, however,  appear to be insufficient in fully measuring it's capabilities. They rely mostly on syntactic accuracy, overlooking the nuanced semantic value of the generated texts \parencite{bubeck_sparks_2023}.  The evolution of GPT, combined with the widespread use of its free chatbot application, ChatGPT, and debate about it's intelligence, or lack of it, has garnered substantial public attention, and the novelty effects on its applications and public reactions can't be underestimated.



% From the approach


%Furthermore, debates have arisen regarding the adequacy of our previous approaches to digital technologies in understanding Human-AI interactions, both in general use and in learning environments . One of such approaches centers on the research around \textit{Computational Thinking}. Originally intended to describe specific competencies in programming, its definition has since evolved and sparked debate. While some regard it as a universal skill useful for numerous tasks in our digital society \parencite{wing_computational_2006}, there is ongoing discussion on its definition and measurements \parencite{MorenoLen2018OnCT} . Additionally, the underlying nature of AI has triggered major changes in computation that demands for an updated understanding for the Computational Thinking framework, particularly in the educational landscape due to its current influence in curriculum design\parencite{tedre_ct_2021}. To clarify the debate and deepen our knowledge in digital technologies, new studies should explore the relations between such frameworks and the most recent technological tools within grounded educational interventions. 

However, as seen in Phenomena-based or Problem-based learning methodologies, open-ended objectives can challenge students due to the lack of guidance or pre-defined paths to follow.


\item[] \textbf{RQ2.} What kind of educational research do educational specialists devise for the application of Artificial Intelligence in education?
    \begin{enumerate}
        \item[] Alternatives
        \begin{enumerate}
            \item What are the predominant themes or focuses in the research envisioned by educational specialists regarding Artificial Intelligence in education? 
        \end{enumerate}
    \end{enumerate}
    \item [] \textbf{RQ3.} What kind of pedagogical value, or lack of it, do educational specialist envision in the development and integration of AI in education?
    \begin{enumerate}
        \item [] Alternatives
        \begin{enumerate}
            \item To what extent do educational specialists foresee AI influencing pedagogical outcomes in educational contexts?
            \item What are the anticipated pedagogical benefits and challenges of AI integration as envisioned by educational specialists? 
        \end{enumerate}
    \end{enumerate}
