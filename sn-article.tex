%Version 2.1 April 2023
% See section 11 of the User Manual for version history
%
%%%%%%%%%%%%%%%%%%%%%%%%%%%%%%%%%%%%%%%%%%%%%%%%%%%%%%%%%%%%%%%%%%%%%%
%%                                                                 %%
%% Please do not use \input{...} to include other tex files.       %%
%% Submit your LaTeX manuscript as one .tex document.              %%
%%                                                                 %%
%% All additional figures and files should be attached             %%
%% separately and not embedded in the \TeX\ document itself.       %%
%%                                                                 %%
%%%%%%%%%%%%%%%%%%%%%%%%%%%%%%%%%%%%%%%%%%%%%%%%%%%%%%%%%%%%%%%%%%%%%

%%\documentclass[referee,sn-basic]{sn-jnl}% referee option is meant for double line spacing

%%=======================================================%%
%% to print line numbers in the margin use lineno option %%
%%=======================================================%%

%%\documentclass[lineno,sn-basic]{sn-jnl}% Basic Springer Nature Reference Style/Chemistry Reference Style

%%======================================================%%
%% to compile with pdflatex/xelatex use pdflatex option %%
%%======================================================%%

%%\documentclass[pdflatex,sn-basic]{sn-jnl}% Basic Springer Nature Reference Style/Chemistry Reference Style


%%Note: the following reference styles support Namedate and Numbered referencing. By default the style follows the most common style. To switch between the options you can add or remove “Numbered” in the optional parenthesis. 
%%The option is available for: sn-basic.bst, sn-vancouver.bst, sn-chicago.bst, sn-mathphys.bst. %  
 
%%\documentclass[sn-nature]{sn-jnl}% Style for submissions to Nature Portfolio journals
%%\documentclass[sn-basic]{sn-jnl}% Basic Springer Nature Reference Style/Chemistry Reference Style
\documentclass[sn-mathphys, Numbered]{sn-jnl}% Math and Physical Sciences Reference Style
%%\documentclass[sn-aps]{sn-jnl}% American Physical Society (APS) Reference Style
%%\documentclass[sn-vancouver,Numbered]{sn-jnl}% Vancouver Reference Style
%%\documentclass[sn-apa]{sn-jnl}% APA Reference Style 
%%\documentclass[sn-chicago]{sn-jnl}% Chicago-based Humanities Reference Style
%%\documentclass[default]{sn-jnl}% Default
%%\documentclass[default,iicol]{sn-jnl}% Default with double column layout

%%%% Standard Packages
%%<additional latex packages if required can be included here>

\usepackage{graphicx}%
\graphicspath{ {./Figures/} }
\usepackage{multirow}%
\usepackage{amsmath,amssymb,amsfonts}%
\usepackage{amsthm}%
\usepackage{mathrsfs}%
\usepackage[title]{appendix}%
\usepackage{xcolor}%
\usepackage{textcomp}%
\usepackage{manyfoot}%
\usepackage{booktabs}%
\usepackage{algorithm}%
\usepackage{algorithmicx}%
\usepackage{algpseudocode}%
\usepackage{listings}%
\usepackage[
style=apa
]{biblatex}%
\addbibresource{myref.bib}
\addbibresource{references.bib}
%%%%

%%%%%=============================================================================%%%%
%%%%  Remarks: This template is provided to aid authors with the preparation
%%%%  of original research articles intended for submission to journals published 
%%%%  by Springer Nature. The guidance has been prepared in partnership with 
%%%%  production teams to conform to Springer Nature technical requirements. 
%%%%  Editorial and presentation requirements differ among journal portfolios and 
%%%%  research disciplines. You may find sections in this template are irrelevant 
%%%%  to your work and are empowered to omit any such section if allowed by the 
%%%%  journal you intend to submit to. The submission guidelines and policies 
%%%%  of the journal take precedence. A detailed User Manual is available in the 
%%%%  template package for technical guidance.
%%%%%=============================================================================%%%%

%\jyear{2023}%

%% as per the requirement new theorem styles can be included as shown below
\theoremstyle{thmstyleone}%
\newtheorem{theorem}{Theorem}%  meant for continuous numbers
%%\newtheorem{theorem}{Theorem}[section]% meant for sectionwise numbers
%% optional argument [theorem] produces theorem numbering sequence instead of independent numbers for Proposition
\newtheorem{proposition}[theorem]{Proposition}% 
%%\newtheorem{proposition}{Proposition}% to get separate numbers for theorem and proposition etc.

\theoremstyle{thmstyletwo}%
\newtheorem{example}{Example}%
\newtheorem{remark}{Remark}%

\theoremstyle{thmstylethree}%
\newtheorem{definition}{Definition}%

\raggedbottom
%%\unnumbered% uncomment this for unnumbered level heads

\begin{document}

\title[Article Title]{Exploring the Use of GPT-4  in Co-creating Personalized Case Scenarios for Higher Education. }

%%=============================================================%%
%% Prefix	-> \pfx{Dr}
%% GivenName	-> \fnm{Joergen W.}
%% Particle	-> \spfx{van der} -> surname prefix
%% FamilyName	-> \sur{Ploeg}
%% Suffix	-> \sfx{IV}
%% NatureName	-> \tanm{Poet Laureate} -> Title after name
%% Degrees	-> \dgr{MSc, PhD}
%% \author*[1,2]{\pfx{Dr} \fnm{Joergen W.} \spfx{van der} \sur{Ploeg} \sfx{IV} \tanm{Poet Laureate} 
%%                 \dgr{MSc, PhD}}\email{iauthor@gmail.com}
%%=============================================================%%

\author*[1]{\fnm{Pablo} \sur{Flores}}\email{pablo.flores@helsinki.fi}

\author[1]{\fnm{Guang} \sur{Rong}}\email{guang.rong@helsinki.fi }

\author[1,2]{\fnm{Benjamin Ultan} \sur{Cowley}}\email{ben.cowley@helsinki.fi}


\affil*[1]{\orgdiv{Faculty of Educational Sciences}, \orgname{University of Helsinki}, \orgaddress{\street{Siltavuorenperger 5}, \city{Helsinki}, \postcode{00014 },  \country{Finland}}}
\affil*[2]{\orgdiv{Cognitive Science, Faculty of Arts}, \orgname{University of Helsinki}}


%%==================================%%
%% sample for unstructured abstract %%
%%==================================%%

\abstract{In the fast evolving landscape of Artificial Intelligence in Education (AIEd) Large Language Models (LLMs) like GPT-4 have emerged as powerful and versatile tools capable of  adapting for a wide set of natural language tasks. This study explores into an interaction design where higher education students 
%Consider changing the "HE students" for Educational Researchers
utilized GPT-4 to craft personalized case scenarios tailored for their coursework. This process not only facilitates tailored learning experiences, but also engaged students into speculative reflections about the potential futures of AIEd  ...................... Add up on results and conclusions when done}

%%================================%%
%% Sample for structured abstract %%
%%================================%%

\keywords{ChatGPT, Large Language Models, Higher Education, Human-AI Co-creation ,Personalized Learning, Speculative Scenarios}

%%\pacs[JEL Classification]{D8, H51}

%%\pacs[MSC Classification]{35A01, 65L10, 65L12, 65L20, 65L70}

\maketitle

\section{Introduction}

% Digital tech in education - reshaping the landscape.
Upon every technological breakthrough, the educational domain reconsiders its methods and strategies. During the recent decades, the rise of digital technologies has reshaped the educational landscape. Evidence from the past 40 years on educational technologies impact consistently indicates positive benefits from its integration in programming courses or through innovative approaches \parencite{higgins_impact_2012}.  
%This digital evolution has fostered the integration of programming and technology courses in schools, as well as innovative learning approaches, such as maker education \parencite{blikstein2013digital}.
However, while digital technologies might spark motivation and engagement in young students, its mere adoption does not ensure favorable outcomes.  To fully embrace their potential, a deliberate pedagogical approach is essential, especially when transitioning from traditional to modern educational approaches \parencite{parker_authentic_2020,khaddage_bridging_2021} . 

% AIEd - Extensive, but limited, integration of recent new technologies
As part of the larger digital transformation, Artificial-Intelligence (AI)-based systems have emerged as important players in the educational domain. These systems have found extensive use in administration, instruction, and learning \parencite{chen_application_2020}, accompanied by a growing body of research around AI in Education (AIEd), a distinct sub-field of digital learning research \parencite{niemi_ai_2023}.  But despite the wide range of potential applications of AI-based systems, they have been mostly implemented to facilitate, or automate, mainstream learning approaches. While useful, this approach sidelines teachers agency, experience, and creativity to integrate such technologies into their unique practices and pedagogical designs. \parencite{holmes_artificial_2023}

%  Dominant trend of STEM design in AIEd
A dominant trend has been that AIEd implementations are predominantly designed by technical teams or departments, such as those focused in Science, Technology, Engineering, and Math (STEM), leading to a notable under-representation of educational or psychological perspectives in AIEd research \parencite{holmes_state_2022, zawacki-richter_systematic_2019}. Among the probable causes of this we can highlight the lack of technical expertise, among others, which is a big barrier for educational adoption of technologies \parencite{reid_categories_2014}. However, as with other digital technologies, AI systems are evolving to become more accessible by reducing their technical entry barriers, both for casual use and for the design of new applications.

%  LLMs - Recent developments offer incredible opportunities
One of such evolution are Large Language Models (LLMs) such as OpenAI's Generative Pre-Trained (GPT) model. They have significantly evolved into systems capable of performing various natural language tasks \parencite{brown_language_2020}. Interestingly, they have shown emergent features as they are able to perform tasks that were not originally part, or expected, of the design \parencite{wei_emergent_2022}. For example, GPT-3 could adapt to new tasks using an approach known as \textit{In-context learning}---relying on natural language descriptions of the task---an ability that it's predecessor struggles to perform consistently \parencite{brown_language_2020, wei_emergent_2022}. This adaptation for different tasks, without requiring exhaustive re-design of their architectures, combined with their potential to serve as building blocks for task-specific AI-tools has led to some researchers to classify them as foundational models \parencite{eloundou_gpts_2023, bommasani_opportunities_2022}.  Furthermore, its last iteration (GPT-4) brought a significant increase in performance across every measured metric \parencite{openai2023gpt4}.
These capabilities present wide opportunities in both design and application. Given that state-of-the-art models do not require advanced technical programming skills, professionals from different domains might now tailor customized tools that align closely with their contexts, needs, and specialized approaches \parencite{cain_gpteammate_2023}. Building on this accessibility, OpenAI has recently announced the upcoming "GPT builder", which will enable users to tailor specific web-based GPT applications for private or shared use \parencite{openaidevshowcase, openai_introducing_2023}. Altogether, a new generation of commercial, public, and private LLMs-based AI applications is on the horizon, with many speculated to be significant in shaping the near to mid-term future \parencite{bommasani_opportunities_2022, bubeck_sparks_2023}. 

% P4 - ChatGPT - Free and widespread chatbot application of GPT-4. Conversational dynamic
Due to its versatility and accessibility, OpenAI's "ChatGPT" has become the fastest-growing application in history since its launch on late 2022 \parencite{milmo_chatgpt_2023}. As a conversational chatbot, it uses natural language processing to understand and generate human-like text in a dialectical fashion. By providing certain input (prompts) ChatGPT can answer different kinds of questions. Additionally, by using advanced prompting methods users can improve its performance in a wide range of tasks \parencite{wei_chain--thought_2023,fernando_promptbreeder_2023}. Potential educational uses involve teaching preparation (generation of course materials, providing suggestions); Assessment (generation of exercises or case scenarios, providing feedback); Learning support (answering questions, summarising information) among others \parencite{lo_what_2023,montenegro-rueda_impact_2023}.  However, educational research remains sparse and focuses predominantly in theoretically exploring its potential and limitations \parencite{qadir_engineering_2022,cain_gpteammate_2023}, and assessing its performance on traditional assessment methods \parencite{nisar_is_2023}. Due to its novelty, a gap in exploratory empirical studies is evident.

We focused our study on exploring the interaction between students and GPT-4 for the creation of personalized course materials in a higher education doctoral course. Our research not only aims to pave the way for innovative designs that enrich and personalize in-person learning experiences using AI tools but also to inspire deeper exploration into the vast potentials of integrating AI in educational settings. 




\subsection*{Approach}\label{Approach}  

%This study focuses on exploring the interaction between students and LLMs, in this case GPT-4, for the co-creation of personalized case scenarios in a Higher Education course with the aim to analyze it from an Information foraging approach and also seeing how Computational Thinking Skills may modulate this behavior

% The context higher education course regarding AI in education
The study was conducted in an on-campus doctoral course titled “Basics of AI in education”, aimed at exploring historical and contemporary AIEd developments. It covered an overview of the historical technical evolution of AIEd systems, examination of current popular systems, like performance predictors, AI-tutors and LLMs, a review of the intersection between AI and cognitive sciences, as well as a discussion of emerging ethical concerns and regulatory developments. 

% Purpose of the course - Speculative assigments - flexible topics
We designed the course assignments to explore hypothetical scenarios about AI implementations in education. For this purpose the students were asked to individually design a study, and write a reflective essay about a specific chosen scenario. Consequently, the course topics were broad and suggestive, not focused on any pre-defined scenario, giving the students freedom to choose a scenario of their own interest. 
Complementing the course contents with a practical experience, we guided the students to interactively generate their hypothetical scenarios with the support of AI. 


To facilitate this, we designed a guided interaction to co-create personalized hypothetical scenarios through GPT-4, which involved pre-defined \textit{prompt templates} and a \textit{set of keywords} to construct scenario generation prompts. We guided our design to address two main objectives.

% Study/design purpose - To examine the behavioral aspects of students-GPT interactions
Firstly, to accommodate students with diverse levels of experience with ChatGPT. Due to its novelty, we had to guide students without experience in interacting with ChatGPT before. By providing clear guidelines and a set of actions to choose from, outlined by prompt templates, we expected that students lacking prior experience could complete their co-creation task without major technical challenges.
Secondly, to enable a systematic and reproducible examination of interactions with ChatGPT by establishing a controlled interaction protocol. Not only it standardizes how students interact with ChatGPT but it can be tailored to capture specific elements according to our research interests, particularly the behavioral dimensions of students interaction.
% Information Foraging

Upon reflecting on the conversational dynamics of ChatGPT in the context of our scenario co-creation task, we opted to guide our prompt design with insights from behavioral models used in Information Foraging Theory. Given that ChatGPT has contextual memory of the conversation, it enables not only to generate scenarios but also to delve deeper by prompting for additional details, thus enriching the scenario with more information. This dynamic closely mirrors two core aspects of Information Foraging Theory: \textit{Exploration}, where new information landscapes are searched (scenarios), and \textit{Exploitation}, where agents decide to utilize a landscape to its fullest potential---in this case, further elaborating an interesting scenario.\parencite{todd_foraging_2020,hills_exploration_2015,cohen_should_2007}. Therefore we shape our prompt design around these distinct foraging actions, conceptualizing the interactions as a a decision-making process within an information foraging task.  % Should I add here smt like " Note that, contrary to prompt engineering studies, we are designing our prompt focusing in examining human behavior, instead of the LLM performance."

% Computational Thinking
Furthermore, we deemed worthy to focus our analysis in examining the role of Computational Thinking Skills in modulating students interactions with ChatGPT. In a seminal contribution, Wing \parencite*{wing_computational_2006} underscored the importance of Computational Thinking skills, which enables people to solve tasks in a similar way that computer algorithms work and are beneficial across most professional domains. Since then, Computational Thinking Skills have grown into a cornerstone of research in digital education and different researchers argue that they correlate with more confident and efficient use of digital technologies \parencite{cansu_overview_2019,grover_computational_2013, shute_demystifying_2017}.Regarding the influence of Computational Thinking over Human-AI Interactions, Celik \parencite*{celik_exploring_2023} explored the determinants of AI literacy, which encompass the knowledge for using, recognizing and evaluating AI-based tools, and reported a significant correlation with Computational Thinking Skills. 

Building on these contributions, we aim to delve deeper into the practical manifestations of such association and investigate whether the unique nature of LLMs, distinct from other AI systems, might reveal new insights into Computational Thinking Skills within the evolving landscape of education. 

%It also aims to engage educational specialists in speculative reflection about the future of AI in education. Our research not only aims to pave the way for innovative designs that enrich and personalize physical learning experiences using AI tools but also to inspire deeper exploration into the vast potentials of integrating AI in educational settings. 

%\subsection{Speculative Methods}\label{Speculative Methods}
%What it is :Development in social sciences and extension
%Extension towards digital education research
%Why for us: Educational specialists views on the future gives insight on the current future-paving processes.

Therefore, in our study we focused our analysis in exploring the following aspects.
\begin{enumerate}
    \item[] \textbf{RQ1.} What are the characteristics of students'-AI behavior within a guided interaction with ChatGPT-4 in terms of Exploration-Exploitation decisions?
    \item []
    \item  [] \textbf{RQ2.} How do the students' Computational Thinking Skills influence their information foraging behavior when co-creating scenarios with ChatGPT-4?
              \begin{enumerate}
              \item[] Alternatives:
        \begin{enumerate}
            \item  Through which behaviors do students showcase their Computational Thinking skills during their engagement with GPT-4?
            \item How are students' Computational Thinking skills reflected in their interactions with GPT-4? 
            \item In what ways do students demonstrate Computational Thinking skills when engaging with GPT-4 in a structured setting?

    \end{enumerate}
 \end{enumerate}
 
\end{enumerate}

\textit{Note, alternatives chosen by a set of alternatives co-created with GPT-4 }


\section{Methods}\label{Methods}

This section describes our participants, the interaction design, and the conducted Computational Thinking Scale. The study was carried out in accordance to the code of ethics of blablabla. Informed consent was obtained from all participants.

\subsection*{Participants}

We conducted this study within an international doctoral course called "Basics on Artificial Intelligence in Educational Sciences" at the University of Helsinki. Initially, 16 students registered to the course, but only 12 completed it. We invited these students to join the study and 10 agreed to participate. At the first class, we presented the main author to the students and the purpose of studying the development of the course; however, they were naive to the study's specific aims.

Participants ranged in age from 30 to 40 (mean = 32.8), with near equal gender distribution (5 females). Six of them were Finnish and the rest a combination of Chinese and Koreans, they all had fluent English skills. They specialized in different educational fields of research, such as Early Childhood Education, Educational Psychology, Science Education, and Higher Education. As an optional course, we assume that all participants had previous interests towards AI applications in education, either for personal or research motivations. Five of them reported previous experiences in using ChatGPT for work, research or personal experiments. The other four reported no previous experience with ChatGPT, or similar applications. 

Considering the small and non-representative nature of our sample, we focus our study on describing our localized experience and generating hypothesis rather than generalizing findings for broader populations.  


%The interaction and survey data was collected from ten participants who enrolled into "Basics on AI in Education" doctoral course at the University of Helsinki. Eight of them were doctoral students and two master students, all of them involved in educational research. A majority of them were Finnish, but the course activities were done in English as the course is part of the international doctoral programme. Their previous experiences with ChatGPT, or AI, differed. While most had extensive experience and were constantly using ChatGPT in their workflow, others had not interacted with ChatGPT before the course.



\subsection*{Guided Interaction Design}\label{Interaction design}

As mentioned before, our design aims to orient the participants during the task and enable a methodical analysis of the interaction. It consists of six modifiable \textit{prompt templates}, divided in two types; a set of \textit{keyword cards}, individual cards with relevant concepts; and an \textit{interaction protocol}, outlining each interaction process. With them, participants crafted a diversity of prompts tailored to their interests, which were used to create and elaborate their hypothetical scenarios in ChatGPT. 

\subsubsection*{\textit{Prompt Templates and Set of Keywords}}

We created the\textit{ prompt templates} by distinguishing two main categories, following the information foraging theoretical approach described above. 

Firstly, we created three different \textit{Exploration prompts} aimed at creating unique scenarios. Their nature was defined by their start ("Describe to me a scenario..." and "Predict a future scenario...") and they allowed for the combination of a maximum of 6 keywords, which granted sufficient conceptual context for our co-creation aims. They provided the starting point to create and explore different scenarios and informational landscapes.

Secondly, we created three \textit{Exploitation prompts} to further elaborate already generated scenarios. They prompted for more information about the specified scenario, focusing in one, or two, keywords of interest. Therefore, they enabled the exploitation of an interesting scenario for more information.

This distinction between the two types of \textit{prompt templates} facilitates our analysis by enabling us to clearly categorize their use within the interaction as either \textit{exploration} or \textit{exploitation} actions.

To construct the \textit{set of keywords}, we envisioned seven categories that generally described AI-influenced educational scenarios. These categories were: Effects, Actors, AI Tools, Educational Process, Environment, Subject, and Activity. Afterwards, aiming of examining the participants' personal interests in driving the interaction, these categories were populated based on an initial course assignment, in which participants were asked to introduce themselves and their interests in relation to the aforementioned categories. Additionally, by taking into account the role of their reported interests, this fostered a student-driven approach that might enhance the relevance and engagement of the participants within the task. As a result, we created a set of 86 keywords distributed unevenly in the different categories. They provided the conceptual context of every generated scenario, according to the participants' choices and combinations of them.

Furthermore, we allowed the participants to use keywords provided by ChatGPT. After observing the nature of its answers to our prompts, where ChatGPT generally offered bullet points to describe the main aspects of the generated scenario, we found interesting to allow the use of its points headers as keywords. These \textit{GPT-sourced keywords} enabled more flexibility and precision, particularly if participants were prone to exploit an interesting scenario based on its specific conceptual information, which would not be necessarily present in the \textit{keywords set}. This added another interesting layer of theoretical depth, as we could observe how the \textit{Exploration} and \textit{Exploitation} of information intertwined with both human-generated concepts and AI-generated concepts.

Both the \textit{prompt templates} and the \textit{keywords} were later printed and laminated. The \textit{prompt templates} consisted of a fixed text written in black representing unchangeable elements, along with variable spaces written in colored texts. These colored spaces corresponded to the different categories where participants could select and combine various concepts given by the \textit{set of keywords.} The keywords were represented as a series of color-coded cards, each card corresponding to an individual keyword and colored by category. The color-coding matched the ones used for the \textit{prompt templates}, to enable a more intuitive understanding of the allowed combinations.

To facilitate the interaction process, we designed an \textit{interaction protocol} to guide the use of the \textit{prompt templates} with the \textit{set of keywords}. This protocol guided the process, standardized my role as a researcher, and constrained my influence over the participants' decision-making process. This approach ensured sufficient space for participants' agency in driving the interaction, while maintaining a helpful guidance throughout the interaction.

\subsubsection*{\textit{Interaction Protocol}}

We scheduled individual sessions with each student for the task. At the time, access to ChatGPT-4 required premium subscription. Consequently, students conducted their task through the author's computer. The data from these interactions was initially stored in author's ChatGPT account. After participants signed the consents, we extracted the data for analysis and deleted any data regarding non-participants.

As the main author, I facilitated the sessions, guiding the participants over the co-creation process according to the following interaction protocol:
    \begin{itemize}

        \item  Displaying the Prompt templates on a table and distributing the keywords cards, face down, almost equally in 5 rows, divided by the coloured categories.
        \item I reiterated the aim of the task---creating an interesting scenario for their assignments---and introduced the use of prompt templates and the keywords, including an example to illustrate their application. I continuously reminded their uses during the interaction to ensure they could effectively engage with the task.
        \item When the aim and rules were clear, participants began by uncovering the first row of cards, selecting a template and keywords of interest to craft a prompt. I then inputted the prompt into ChatGPT and displayed the generated text. Participants were then reminded of their different options: to generate (\textit{explore}) new scenarios, further elaborate (\textit{exploit}) the current one, or concluding the task by choosing the scenario. I also highlighted the option of using GPT-sourced keywords to explore or exploit scenarios.
        \item Once the participant decided that a scenario was interesting enough for them to end the task, I provided them with the full transcript of the conversation for the selected scenario. They then proceeded to work on their assignments, focusing on the chosen scenario.
    \end{itemize}


\subsection*{Computational Thinking Scale}

Following our purpose of examining the role of Computational Thinking within our interaction process, and following Celik's previous study on AI literacy \parencite*{celik_exploring_2023}, we chose to employ the same instrument, the Computational Thinking Scale (CTS) survey.

Developed by Korkmaz et al. \parencite*{korkmaz_validity_2017-1}, it is a self-report scale based on the framework by CSTA \& ISTE \parencite*{csta_iste_operational_2015}, and the Computational Thinking Leadership Toolkit ISTE \parencite*{iste_computational_2011}. They define computational thinking as the result of 5 intertwined skills, Creativity, Algorithmic Thinking, Cooperativity, Critical Thinking, and Problem Solving. Consequently, the scale consists of 29 items divided in the corresponding 5 sub-factors. Based on a five-point Likert-type rating structure, with higher scores indicating greater development of computational thinking skills. 

Additionally, considering that the scale was developed in computer sciences learning environments we adapted three items, aiming for more contextual clarity in grammar and semantics. We adapted mathematically specific terms to be more generally understood. We deemed this adaptations suitable for our distinct educational environment.

%Add CTS reliability calculat

\subsection*{Data Analysis}

Initially, we processed the behavioral data from interactions using the ATLAS.ti qualitative analysis software tool. A frequency-based data set was constructed based on the interactions' logs provided by ChatGPT. By crossing the logs with the prompt templates, set of keywords, and students' initial interests (reported during first assignment), we defined 8 variables describing the interaction. Both the Computational Thinking Scale and the interaction data were then analyzed within R statistical computing environment.

First, for RQ1 (regarding the interaction characterization) we used descriptive analysis to examine participants' behavioral data. Following the available choices provided by the interaction design, we gathered 5 distinct discrete variables describing: Information foraging actions (Exploration and Exploitation); Total flipped keywords cards; and the source of the used cards (either provided by the guide set of keywords cards, or by the GPT-generated scenarios). Additionally, we classified the used keywords according to the first introductory course assignment, where students provided their initial interests to construct the set of keywords. As such, we distinguished between used of keywords matching their initially reported interests or not. Looking to surface meaningful associations. To surface meaningful associations and mitigate the risk of false negatives, we adopted a grammatically and semantically flexible classification approach. This was essential because some keywords, such as 'Higher-Education' and 'University,' represented closely related interests. 

Lastly, we extracted the conversation time length from the conversation logs. This measured the time between the first and the last prompt, not accounting for the task introduction and first prompt construction time. 



To investigate RQ2, regarding the potential associations between participants' interactions and their Computational Thinking skills, we employed network analysis methods.

We estimated the network structure by first calculating a statistical model from the data and then analyzing the weighted network measures, estimating their stability with bootstrapping methods \parencite{kotz_bootstrap_1992, hevey_network_2018}. These procedures were conducted using the \textit{bootnet} package \parencite{epskamp_estimating_2018}

Accounting for the non-normal distribution, low sample size, and the mix of ordinal and continuous variables in our data, we estimated a correlation network using Kendall's correlation matrix \parencite*{kendall_rank_1949}.  To minimize spurious edges, we set a standard statistical significance threshold of $p < 0.05$ and adjusted the correlation strength with a lower threshold of $\tau > 0.4$, based on Dancey \& Reidy's \parencite*{dancey_statistics_2007} correlations' strength interpretation in psychological studies.
This approach enabled us to produce a denser network (with more edges), which is more suitable for initial data exploration and hypothesis generation in our study context.

Network visualization was possible using \textit{qgraph} package \parencite{epskamp_qgraph_2012}, with colorblind-friendly colors (blue for positive and red for negative edge weights) and nodes distinguished by their nature (blue for CTS and orange for behavioral variables).

To estimate edge weights, we used non-parametric bootstrap (resampling with replacement) to create 2500 samples and estimate edge weights stability. To estimate centrality indices stability, we used case-dropping subset bootstrap samples (n = 1000). Correlation Stability coefficient for correlation values were used to measure stability of centrality indices \parencite{epskamp_estimating_2018}. CS-coefficient indicates the percentage of our sample that can be dropped to maintain, with a 95\% confidence interval, correlation values equal or above to r = 0.7 between our sample’s centrality indices and our bootstrapped samples' centrality indices.



\section{Results}


\begin{table}[ht] \caption{Descriptive statistics of gathered data.}
\centering
\begin{tabular}{lrrrrrrrrr}
  \hline
  vars & & n & mean & sd & min & max & range & se & type \\
  \hline
   1&Exploration & 9.00 & 1.56 & 0.53 & 1.00 & 2.00 & 1.00 & 0.18 & Discrete \\
   2&Exploitation & 9.00 & 2.22 & 1.56 & 0.00 & 5.00 & 5.00 & 0.52 & Discrete \\
   3&Flipped\_cards & 9.00 & 1.89& 0.60 & 1.00& 3.00& 2.00 & 0.20 & Discrete \\
   4&GPT\_sourced\_Kw & 9.00 & 1.56 & 1.94 & 0.00 & 6.00 & 6.00 & 0.65 & Discrete \\
   5&Guide\_sourced\_Kw & 9.00 & 8.33 & 2.65 & 5.00 & 12.00 & 7.00 & 0.88 & Discrete \\
   6&Reported\_interest\_Kw & 9.00 & 2.56 & 2.40 & 0.00 & 7.00 & 7.00 & 0.80 & Discrete \\
   7&Nonreported\_interest\_Kw & 9.00 & 7.33 & 2.35 & 4.00 & 11.00 & 7.00 & 0.78 & Discrete \\
   8&Conversation.Duration [min] & 9.00 & 14.79 & 8.58 & 4.03 & 30.22 & 26.18 & 2.86 & Continuous \\
   9&Creativity & 9.00 & 29.11 & 3.18 & 24.00 & 32.00 & 8.00 & 1.06 & Ordinal \\
   10&Critical\_thinking & 9.00 & 18.89 & 3.89 & 13.00 & 25.00 & 12.00 & 1.30 & Ordinal \\
   11&Algorithmic\_thinking & 9.00 & 16.78 & 4.02 & 11.00 & 24.00 & 13.00 & 1.34 & Ordinal \\
   12&Cooperativity & 9.00 & 15.33 & 2.78 & 11.00 & 20.00 & 9.00 & 0.93 & Ordinal \\
   13&Problem\_solving & 9.00 & 20.67 & 2.87 & 15.00 & 24.00 & 9.00 & 0.96 & Ordinal \\
   14&Computational Thinking Scale score & 9.00 & 100.78 & 11.70 & 82.00 & 119.00 & 37.00 & 3.90 & Ordinal \\
   \hline
\end{tabular}
\label{summary}
\end{table}

All participants completed the scenario generation task. However, one did not follow the interaction protocol and was dropped from the analysis, resulting in a sample size of n = 9. Table \ref{summary} shows the descriptive statistics for the interaction (vars 1-8) and the self-reported CTS results (vars 9-14). 

\subsection{RQ1. Describing the interaction model.}

\subsubsection{General observations}
Our results shows that the interaction was somewhat short in duration (m = 14.79 minutes) and prompts (m = 3.78).  The longer interaction involved six prompts and the shorter two, providing a suggestive length for an activity of such nature. By observing the interactions, we could realize that participants were spending most of the time verbally reflecting on the generated scenarios and the novelty of the tool.  



\subsubsection*{Exploration vs Exploitation efforts}


\begin{table}  \caption{Total uses of prompt types and source of involved keywords.}
    \centering
    \begin{tabular}{c|cccc}   
         Prompt type&  Total uses&  Guide-sourced Kw&  GPT-sourced Kw& \\ \hline  
         Exploration&  14&  66&  0& \\   
         Exploitation&  20&  9&  14& \\ 
    \end{tabular}
    \label{tab:Xi vs Xr}
\end{table}


The results show that the total count of exploitation prompts is higher than exploration prompts (Tab\ref{tab:Xi vs Xr}). As shown in Table \ref{summary}, participants employed a maximum of 2 explorations, consequently we can only distinguish between the ones that chose to explore again, and the ones that only explored one scenario. Almost every participant –except from one– executed at least one exploitation action.
% I LEFT IT HERE
No participant stopped at the minimum possible number of actions, generating an scenario and then choosing it without exploiting it or exploring new ones. 

In Fig. 3 we can observe how most participants executed one or two explorations, with one participant exploring 3 different scenarios. In contrast, the participants showed more variability in the number of exploitations, still centered around 2-3 exploitations but with some executing more, or none.

\subsubsection*{Source of Keywords}

\subsubsection*{Keywords of interest}

\subsubsection*{Network Analysis}

Do the simpler one and then expand into the subfactors.

Don't put the stability graphs

\subsubsection*{Accuracy}





\section{Discussion}\label{Discussion}

Discussions should be brief and focused. In some disciplines use of Discussion or `Conclusion' is interchangeable. It is not mandatory to use both. Some journals prefer a section `Results and Discussion' followed by a section `Conclusion'. Please refer to Journal-level guidance for any specific requirements. 


\subsection{Limitations}\label{Limitations}

\subsection{Further Lines of Research}\label{Further Lines of Research}

\subsection{Conclusion}\label{Conclusion}

Conclusions may be used to restate your hypothesis or research question, restate your major findings, explain the relevance and the added value of your work, highlight any limitations of your study, describe future directions for research and recommendations. 

In some disciplines use of Discussion or 'Conclusion' is interchangeable. It is not mandatory to use both. Please refer to Journal-level guidance for any specific requirements. 


\printbibliography
%% if required, the content of .bbl file can be included here once bbl is generated
%%\input sn-article.bbl


\end{document}
