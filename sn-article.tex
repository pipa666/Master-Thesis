%Version 2.1 April 2023
% See section 11 of the User Manual for version history
%
%%%%%%%%%%%%%%%%%%%%%%%%%%%%%%%%%%%%%%%%%%%%%%%%%%%%%%%%%%%%%%%%%%%%%%
%%                                                                 %%
%% Please do not use \input{...} to include other tex files.       %%
%% Submit your LaTeX manuscript as one .tex document.              %%
%%                                                                 %%
%% All additional figures and files should be attached             %%
%% separately and not embedded in the \TeX\ document itself.       %%
%%                                                                 %%
%%%%%%%%%%%%%%%%%%%%%%%%%%%%%%%%%%%%%%%%%%%%%%%%%%%%%%%%%%%%%%%%%%%%%

%%\documentclass[referee,sn-basic]{sn-jnl}% referee option is meant for double line spacing

%%=======================================================%%
%% to print line numbers in the margin use lineno option %%
%%=======================================================%%

%%\documentclass[lineno,sn-basic]{sn-jnl}% Basic Springer Nature Reference Style/Chemistry Reference Style

%%======================================================%%
%% to compile with pdflatex/xelatex use pdflatex option %%
%%======================================================%%

%%\documentclass[pdflatex,sn-basic]{sn-jnl}% Basic Springer Nature Reference Style/Chemistry Reference Style


%%Note: the following reference styles support Namedate and Numbered referencing. By default the style follows the most common style. To switch between the options you can add or remove “Numbered” in the optional parenthesis. 
%%The option is available for: sn-basic.bst, sn-vancouver.bst, sn-chicago.bst, sn-mathphys.bst. %  
 
%%\documentclass[sn-nature]{sn-jnl}% Style for submissions to Nature Portfolio journals
%%\documentclass[sn-basic]{sn-jnl}% Basic Springer Nature Reference Style/Chemistry Reference Style
\documentclass[sn-mathphys, Numbered]{sn-jnl}% Math and Physical Sciences Reference Style
%%\documentclass[sn-aps]{sn-jnl}% American Physical Society (APS) Reference Style
%%\documentclass[sn-vancouver,Numbered]{sn-jnl}% Vancouver Reference Style
%%\documentclass[sn-apa]{sn-jnl}% APA Reference Style 
%%\documentclass[sn-chicago]{sn-jnl}% Chicago-based Humanities Reference Style
%%\documentclass[default]{sn-jnl}% Default
%%\documentclass[default,iicol]{sn-jnl}% Default with double column layout

%%%% Standard Packages
%%<additional latex packages if required can be included here>

\usepackage{graphicx}%
\usepackage{multirow}%
\usepackage{amsmath,amssymb,amsfonts}%
\usepackage{amsthm}%
\usepackage{mathrsfs}%
\usepackage[title]{appendix}%
\usepackage{xcolor}%
\usepackage{textcomp}%
\usepackage{manyfoot}%
\usepackage{booktabs}%
\usepackage{algorithm}%
\usepackage{algorithmicx}%
\usepackage{algpseudocode}%
\usepackage{listings}%
\usepackage[
style=apa
]{biblatex}%
\addbibresource{myref.bib}
\addbibresource{references.bib}
%%%%

%%%%%=============================================================================%%%%
%%%%  Remarks: This template is provided to aid authors with the preparation
%%%%  of original research articles intended for submission to journals published 
%%%%  by Springer Nature. The guidance has been prepared in partnership with 
%%%%  production teams to conform to Springer Nature technical requirements. 
%%%%  Editorial and presentation requirements differ among journal portfolios and 
%%%%  research disciplines. You may find sections in this template are irrelevant 
%%%%  to your work and are empowered to omit any such section if allowed by the 
%%%%  journal you intend to submit to. The submission guidelines and policies 
%%%%  of the journal take precedence. A detailed User Manual is available in the 
%%%%  template package for technical guidance.
%%%%%=============================================================================%%%%

%\jyear{2023}%

%% as per the requirement new theorem styles can be included as shown below
\theoremstyle{thmstyleone}%
\newtheorem{theorem}{Theorem}%  meant for continuous numbers
%%\newtheorem{theorem}{Theorem}[section]% meant for sectionwise numbers
%% optional argument [theorem] produces theorem numbering sequence instead of independent numbers for Proposition
\newtheorem{proposition}[theorem]{Proposition}% 
%%\newtheorem{proposition}{Proposition}% to get separate numbers for theorem and proposition etc.

\theoremstyle{thmstyletwo}%
\newtheorem{example}{Example}%
\newtheorem{remark}{Remark}%

\theoremstyle{thmstylethree}%
\newtheorem{definition}{Definition}%

\raggedbottom
%%\unnumbered% uncomment this for unnumbered level heads

\begin{document}

\title[Article Title]{Exploring the Use of GPT-4  in Co-creating Personalized Case Scenarios for Higher Education. }

%%=============================================================%%
%% Prefix	-> \pfx{Dr}
%% GivenName	-> \fnm{Joergen W.}
%% Particle	-> \spfx{van der} -> surname prefix
%% FamilyName	-> \sur{Ploeg}
%% Suffix	-> \sfx{IV}
%% NatureName	-> \tanm{Poet Laureate} -> Title after name
%% Degrees	-> \dgr{MSc, PhD}
%% \author*[1,2]{\pfx{Dr} \fnm{Joergen W.} \spfx{van der} \sur{Ploeg} \sfx{IV} \tanm{Poet Laureate} 
%%                 \dgr{MSc, PhD}}\email{iauthor@gmail.com}
%%=============================================================%%

\author*[1]{\fnm{Pablo} \sur{Flores}}\email{pablo.flores@helsinki.fi}

\author[1]{\fnm{Guang} \sur{Rong}}\email{guang.rong@helsinki.fi }

\author[1,2]{\fnm{Benjamin Ultan} \sur{Cowley}}\email{ben.cowley@helsinki.fi}


\affil*[1]{\orgdiv{Faculty of Educational Sciences}, \orgname{University of Helsinki}, \orgaddress{\street{Siltavuorenperger 5}, \city{Helsinki}, \postcode{00014 },  \country{Finland}}}
\affil*[2]{\orgdiv{Cognitive Science, Faculty of Arts}, \orgname{University of Helsinki}}


%%==================================%%
%% sample for unstructured abstract %%
%%==================================%%

\abstract{In the fast evolving landscape of Artificial Intelligence in Education (AIEd) Large Language Models (LLMs) like GPT-4 have emerged as powerful and versatile tools capable of  adapting for a wide set of natural language tasks. This study explores into an interaction design where higher education students 
%Consider changing the "HE students" for Educational Researchers
utilized GPT-4 to craft personalized case scenarios tailored for their coursework. This process not only facilitates tailored learning experiences, but also engaged students into speculative reflections about the potential futures of AIEd  ...................... Add up on results and conclusions when done}

%%================================%%
%% Sample for structured abstract %%
%%================================%%

\keywords{ChatGPT, Large Language Models, Higher Education, Human-AI Co-creation ,Personalized Learning, Speculative Scenarios}

%%\pacs[JEL Classification]{D8, H51}

%%\pacs[MSC Classification]{35A01, 65L10, 65L12, 65L20, 65L70}

\maketitle

\section{Introduction}\label{Introduction}
% Should I put many references, or a single main one is prefered?
%P1 Recent developments in LLMs and emergent properties 
During the last years, Large Language Models (LLMs) such as GPT-4 have significantly evolved into systems capable of performing various natural language tasks \parencite{brown_language_2020}. More interestingly, they have shown emergent features as they are able to perform tasks that were not originally part, or expected, of the design \parencite{wei_emergent_2022}. Their ability to adapt for different tasks without requiring exhaustive re-design of their architectures, combined with their potential to serve as building blocks for task-specific Artificial Intelligence (AI) tools, has led researchers to classify them as a general purpose technology \parencite{eloundou_gpts_2023}. Notably, GPT-3 can adapt to new tasks using an approach known as \textit{In-context learning}---relying on natural language descriptions of the task---an ability that it's predecessor, GPT-2 (with fewer training parameters and data) struggles to perform consistently \parencite{bommasani_opportunities_2022, wei_emergent_2022}. 
%P2 Potential for specific tools design with lower technical entry-level 
These capabilities present wide opportunities in both design and application. Given that state-of-the-art models do not require advanced technical programming skills, professionals from different domains might now tailor customized tools that align closely with their contexts, needs, and specialized approaches \parencite{cain_gpteammate_2023}. Andrew Ng, a leading expert in the field has stated recently that a new breed of prompt-based AI applications might be designed in few hours, even minutes, just by providing natural language descriptions of the tasks \parencite{unknown-author-2023} and this study will exemplifies such developments. Altogether, a new wave of LLMs-based AI applications is on the horizon, with many speculated to be influential in the near to mid future.\parencite{bommasani_opportunities_2022, bubeck_sparks_2023}.

%P3 Digital Technologies in reshaping education and recent AI integration and the current hype is promting different stakeholders to integrate the new tool into their learning processes. 
In the field of education, the rise of Information and Communication Technologies has reshaped and enriched the educational landscape for decades \parencite{higgins_impact_2012}. Digitalization has led to the integration of programming and technology courses in schools, as well as innovative learning approaches, such as maker education \parencite{blikstein2013digital}. More recently, AI-based systems and tools have been extensively integrated in administration, instruction, and learning \parencite{chen_application_2020} and a growing body of research around AI in Education (AIEd) is laying the foundations for a new specific sub-field of AI research \parencite{niemi_ai_2023}.  Given these advancements, educators, administrators, entrepreneurs, and policy-makers are poised to further integrate emerging technologies into their pedagogical practices, shaping the future educational landscape.

%P4 The mere presence of technology doesn’t guarantee favorable outcomes, there needs to be a thoughtful design  
Although technology offers many tools and opportunities, its mere presence does not assure enhanced educational experiences. Thoughtful integration is essential, as research of digital technologies in education indicates that their effectiveness is often tied more to the quality of design and implementation, rather than just availability for use \parencite{higgins_impact_2012}. Similar observations have been made for interventions involving AI \parencite{holmes_state_2022}. A deeper understanding of student-technology interactions its necessary to guide the design of improved educational tools and foster better learning experiences. 

% P5 There is a lack of empirical research in LLMs use in education 
Unfortunately, empirical studies on LLMs' impact in educational environments remain sparse \parencite{montenegro-rueda_impact_2023, lo_what_2023}. Since the launch of "ChatGPT" (based on GPT-3) earlier this year, there has been a rapid growth in reported educational uses and impacts, though they remain somewhat limited in scope. Predominantly, existing research either evaluates GPT's performance on traditional assessment methods \parencite{nisar_is_2023}  or delves into theoretical exploration of its potential opportunities and challenges \parencite{cain_gpteammate_2023} . Only a select few of them have explored empirical outcomes of granting students unrestricted use of LLMs in their learning process, with the results appearing optimistic \parencite{fauzi_analysing_2023, yilmaz_effect_2023}. More broadly, literature review in AIEd  underscores the necessity for more hands-on educational research approaches and a wider array of analytical methods and data gathering techniques \parencite{zawacki-richter_systematic_2019, holmes_state_2022}. Notably, a significant gap exists within physical educational spaces, as we could not find empirical studies that specifically crafted a methodical design or purposeful approach for LLMs' integration.

\subsection{Approach}\label{Approach}  WHAT - HOW - WHY
In light of the recent surge in LLMs deployment and the scarcity of solid empirical and theoretical insights, we opted to conduct an exploratory empirical study to analyze human-AI interactions within an educational environment.
% P1 Our context - Higher education course regarding AI in education
The study was conducted in an on-campus doctoral course titled “Basics of AI in education”, designed to explore and discuss both historical and contemporary developments of AIEd. It covered an overview of the historical technical evolution of AIEd systems, examination of current popular systems, like AI-tutors and GPT models, a review of the intersection between AI and cognitive sciences, as well as a discussion of emerging ethical concerns and regulatory developments. 

The course assignments were designed to explore hypothetical AIEd scenarios. For this purpose the students where asked to describe a study and write a reflective essay about a specific scenario. Observing the capabilities mentioned above, we found it intriguing to explore a way that the students could craft their hypothetical scenarios with the use of GPT-4. 

With the purpose of gathering empirical data in a controlled fashion, and orienting the students through the use of GPT-4, we designed a specific guided interaction that consisted of a set of prompting templates and keywords to be used. By doing this, we aimed to restrict our research focus on the students behavior within a limited set of actions. To conceive the design we had two guiding frameworks, Information Foraging and Computational thinking

%% I LEFT IT HERE

and how computational thinking skills, with predominancy in digital education research, may influence and modelate the student's behavior.




% P2 Information Foraging as guiding theory
As starting point, we decided to draw on two interesting approaches to explore Human-AI dynamics in education, Computational Thinking and Information Foraging. 

% P3 Computational Thinking as a modulator for the interaction behavior


%Yet, within both the AIEd body of research there is a notable absence of educational specialists representation, a scarcity of qualitative and mixed-methods studies, and a limited variety in data gathering approaches. Particularly relevant is the lack of experiences in physical learning environments and objectives that extend beyond just enhancing students' academic performance \parencite{zawacki-richter_systematic_2019, holmes_state_2022, grover_computational_2013}. 




%PX - Our Focus
This study focuses on exploring the interaction between students and LLMs, in this case GPT-4, for the co-creation of personalized case scenarios in a Higher Education course and its connections with Computational Thinking Skills. It also aims to engage educational specialists in speculative reflection about the future of AI in education. Our research not only aims to pave the way for innovative designs that enrich and personalize physical learning experiences using AI tools but also to inspire deeper exploration into the vast potentials of integrating AI in educational settings. 


%Furthermore, debates have arisen regarding the adequacy of our previous approaches to digital technologies in understanding Human-AI interactions, both in general use and in learning environments . One of such approaches centers on the research around \textit{Computational Thinking}. Originally intended to describe specific competencies in programming, its definition has since evolved and sparked debate. While some regard it as a universal skill useful for numerous tasks in our digital society \parencite{wing_computational_2006}, there is ongoing discussion on its definition and measurements \parencite{MorenoLen2018OnCT} . Additionally, the underlying nature of AI has triggered major changes in computation that demands for an updated understanding for the Computational Thinking framework, particularly in the educational landscape due to its current influence in curriculum design\parencite{tedre_ct_2021}. To clarify the debate and deepen our knowledge in digital technologies, new studies should explore the relations between such frameworks and the most recent technological tools within grounded educational interventions. 




%GPT (Generative Pre-Trained) language models are a particular class of AI systems, distinguished by their capability of generating human-like text from a given prompt. They are able to perform a variety of tasks, from translation to text summarization and question answering, often surpassing than human-average performance \parencite{srivastava_beyond_2022}.  As the volume of training data and trained model parameters has increased, their adaptability to tasks beyond their initial training scope has expanded impressively.  GPT-4 has further improved current capabilities in every measured metric so far \parencite{openai2023gpt4}. Current benchmarks, however,  appear to be insufficient in fully measuring it's capabilities. They rely mostly on syntactic accuracy, overlooking the nuanced semantic value of the generated texts \parencite{bubeck_sparks_2023}.  The evolution of GPT, combined with the widespread use of its free chatbot application, ChatGPT, and debate about it's intelligence, or lack of it, has garnered substantial public attention, and the novelty effects on its applications and public reactions can't be underestimated.


%\subsection{Speculative Methods}\label{Speculative Methods}
%What it is :Development in social sciences and extension
%Extension towards digital education research
%Why for us: Educational specialists views on the future gives insight on the current future-paving processes.

In this study, we explore the gathered behavioral data from the participants' interaction with GPT-4, it's associations with their self-reported Computational Thinking Skills, and their speculations about potential futures in AIEd.
\begin{enumerate}
    \item[] \textbf{RQ1.} How does the students' Computational Thinking Skills influence their co-creation behavioral interactions with GPT-4
    \begin{enumerate}
        \item[] Alternatives:
        \begin{enumerate}
            \item  Through which behaviors do students showcase their Computational Thinking skills during their engagement with GPT-4?
            \item How are students' Computational Thinking skills reflected in their interactions with GPT-4? 
            \item In what ways do students demonstrate Computational Thinking skills when engaging with GPT-4 in a structured setting? 
        \end{enumerate}
    \end{enumerate}
    \item[] \textbf{RQ2.} What kind of educational research do educational specialists devise for the application of Artificial Intelligence in education?
    \begin{enumerate}
        \item[] Alternatives
        \begin{enumerate}
            \item What are the predominant themes or focuses in the research envisioned by educational specialists regarding Artificial Intelligence in education? 
        \end{enumerate}
    \end{enumerate}
    \item [] \textbf{RQ3.} What kind of pedagogical value, or lack of it, do educational specialist envision in the development and integration of AI in education?
    \begin{enumerate}
        \item [] Alternatives
        \begin{enumerate}
            \item To what extent do educational specialists foresee AI influencing pedagogical outcomes in educational contexts?
            \item What are the anticipated pedagogical benefits and challenges of AI integration as envisioned by educational specialists? 
        \end{enumerate}
    \end{enumerate}

\end{enumerate}
\textit{Note, alternatives chosen by a set of alternatives co-created with GPT-4 }



\section{Methods}\label{Methods}

\subsection{Design}\label{Interaction design}
\subsubsection{Prompting templates}
% Templates defined by 2 types of actions and categorization
\subsubsection{Keywords}
%Interests assigment should go here
\subsubsection{Interaction}

\subsection{Quantitative analysis - Network Analysis}\label{Quantitative Analysis}

\subsection{Qualitative analysis- Content Analysis}\label{Qualitative Analysis}



\section{Results}\label{Results}

\section{Discussion}\label{Discussion}

Discussions should be brief and focused. In some disciplines use of Discussion or `Conclusion' is interchangeable. It is not mandatory to use both. Some journals prefer a section `Results and Discussion' followed by a section `Conclusion'. Please refer to Journal-level guidance for any specific requirements. 


\subsection{Limitations}\label{Limitations}

\subsection{Further Lines of Research}\label{Further Lines of Research}

\subsection{Conclusion}\label{Conclusion}

Conclusions may be used to restate your hypothesis or research question, restate your major findings, explain the relevance and the added value of your work, highlight any limitations of your study, describe future directions for research and recommendations. 

In some disciplines use of Discussion or 'Conclusion' is interchangeable. It is not mandatory to use both. Please refer to Journal-level guidance for any specific requirements. 


\printbibliography
%% if required, the content of .bbl file can be included here once bbl is generated
%%\input sn-article.bbl


\end{document}
